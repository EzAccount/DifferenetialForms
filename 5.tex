\documentclass{article}
\usepackage{amsmath,amsthm,amssymb}
\usepackage{amsthm}
\usepackage{mathtext}
\usepackage{mathtools}
\usepackage{graphicx}
\usepackage{wrapfig}
\graphicspath{ {images/} }
\usepackage[T1,T2A]{fontenc}
\usepackage[utf8]{inputenc}
\usepackage[english,russian]{babel}
\usepackage{fancyhdr}
\usepackage[a4paper]{geometry}
\pagestyle{fancy}
\fancyhf{}
\fancyhead[L]{Когомологии де Рама.}
\theoremstyle{definition}
\newtheorem{defn}{Def}
\newtheorem{example}{Ex}
 \newtheorem{theorem}{Th}
 \DeclareMathOperator{\im}{im}
\begin{document}


\centering{
	\subsection*{Листок - 5}}
\begin{enumerate}
	\item Пользуясь ковариантной формой уравнений Максвелла получить уравнения Максвелла в терминах дифференциальных форм. Выписать формы плотности энергии.
	\item Является ли вещественное проективное пространство $\mathbb{RP}^n$ ориентируемым? При каких $n$?
	\item Докажите, что всякое комплексное аналитическое многообразие ориентируемо.
	
	\vspace{3ex}
\textit{Напоминание:} $f$ называется гомоморфизмом групп $(G_1,*),  (G_2, \times )$ если $f(a*b)= f(a) \times f(b)$

\vspace{2ex}
	Последовательность гомоморфизмов называется
	$$
	\dots \longrightarrow A_i \stackrel{\phi_i} {\longrightarrow} A_{i+1} \stackrel{\phi_{i+1}}{\longrightarrow} \dots
	$$ называется полуточной, если $\im \phi_i \subset \ker \phi_{i+1}$ и точной если $\im \phi_i = \ker \phi_{i+1}$ для любого $i$.

\item Рассмотрим последовательность гомоморфизмов:
 	$$
 \dots \longrightarrow \Omega_{p-1} (X) \stackrel{d} {\longrightarrow} \Omega_p(X)  \stackrel{d}{\longrightarrow} \Omega_{p+1} (X) \stackrel{d}{\longrightarrow} \dots
 $$
 Покажите что она всегда полуточна. Приведите пример многообразия для которого эта последовательность точна.
 
 \vspace{3ex}
 Факторпространство $H^P(X) := Z^p(X) / B^p (X)$
 будем называть пространством когомологий де Рама.
 \item Докажите,что $H^n (M) = 0$ если $n > \dim \text M$.
 \item Доказать, что для любого многообразия $X$ когомологии де Рама $H^0(X)\simeq \mathbb{R}^q$, где $q$ - количество связных компонент $X$. Как это согласуется с результатами математического анализа?
 \item Вычислить когомологии де Рама прямой $\mathbb{R}$
 
 \vspace{3ex}
 Область $G \subset \mathbb{R}^n$ будем называть звездной относительно центра $x_0 \in G$, если вместе с каждой точкой $x \in G$ область содержит отрезок $[x_0, x]$
 
 \begin{theorem}[Пуанкаре]
 	Любая замкнутая форма $\omega \in \Omega^p(G)$ в звездной области $G$ точна.
 \end{theorem}
 Назовём отображения $g,f: X \mapsto Y$ гомотопными если существует отображение $F: X \times [0,1] \mapsto Y$, такое что $F(x,0) = f(x)$, $F(x,1) = g(x)$.
 
 Многообразия $X$ и $Y$ называются гомотопически эквивалентными, если существуют гладкие отображение $f: X \mapsto Y$ и $g: Y \mapsto X$, такие, что их композиции гомотопны тождественным отображениям.
 
 \vspace{3ex}
 \item Доказать что гомотопическая эквивалентность сохраняет связность.
 \item Стянуть (показать гомотопическую эквивалентность точке) $\mathbb{R}^n$
 \item Показать что гиперплоскость без точки гомотопически эквивалентна $(n-1)$ мерной сфере.

 \item Найдите когомологии плоскости без одной точки, двумерного тора, двумерной сферы.  \textit{Замечание:} Здесь и далее под когомологиями понимаются когомологии де Рама
 \item Найдите когомологии $\mathbb{RP}^2$.
 \item Докажите что отображение 
 $$ \omega \mapsto \int_{S^n} \omega$$
 задает изоморфизм $H^n(S^n) \simeq R$.
 \item Найдите когомологии $S^n$.


 
 

\end{enumerate}
\vspace{\fill}



\end{document}

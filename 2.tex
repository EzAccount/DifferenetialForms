\documentclass{article}
\usepackage{amsmath,amsthm,amssymb}
\usepackage{mathtext}
\usepackage{mathtools}

\usepackage[T1,T2A]{fontenc}
\usepackage[utf8]{inputenc}
\usepackage[english,russian]{babel}
\usepackage{fancyhdr}
\usepackage[a5paper]{geometry}
\pagestyle{fancy}
\fancyhf{}
\fancyhead[L]{Дифференциальные формы и векторный анализ}
\begin{document}
\subsection*{Некоторые напоминания о декартовой системе} Пусть $\{e_i\}$ - базис. Здесь и далее $(\cdot, \cdot), [\cdot, \cdot]$ - скалярное и векторное произведение соответственно, по повторяющимся индексам предполагается суммирование.
$$(e_i, e_j)=\delta_{ij}$$
$$[e_i, e_j] = \epsilon_{ijk}e_k$$
$$\epsilon_{ijk}\epsilon_{klm} = \delta_{il}\delta_{jm} - \delta_{im}\delta_{jl}$$
$$V=e_i V^i_{(e_i)}$$
Индексная нотация может быть удобна, в частности, для доказательства некоторых утверждений векторного анализа:
\begin{align*}
    [\mathbf{A}, [\mathbf{B},\mathbf{C}]] &= [e_i, [e_l, e_m]] A^i B^l C^m\\
    &= [e_i, e_k] \epsilon_{klm} A^i B^l C^m \\
    &= e_j \epsilon_{jik}\epsilon_{klm} A^i B^l C^m \\
    &= e_j (\delta_{jl}\delta_{im} - \delta_{jm}\delta_{il}) A^i B^l C^m \\
    &= \mathbf{B} (\mathbf{A}\cdot \mathbf{C}) - \mathbf{C} (\mathbf{A}\cdot \mathbf{B})\\
\end{align*}
\subsection*{Криволинейные координаты}
Пусть заданы отображения: $x = x(\zeta_1, \zeta_2, \zeta_3)$,$y = y(\zeta_1, \zeta_2, \zeta_3)$,$z = z(\zeta_1, \zeta_2, \zeta_3)$.
В каждой точке пространства тогда определены вектора:
$$\mathbf{i}_i = \frac{\partial}{\partial \zeta^i} = \mathbf{e_a} \frac{\partial x^a}{\partial \zeta^i}.$$
Точнее в каждой точке $p$ пространства $M$ (которое мы должны, вообще говоря, считать дифференциальным многообразием) определенно соответствующее векторное пространство $\mathbf{T_p M}$, называемое касательным пространством. Вектора $\mathbf{i}_i (\zeta)$ образуют базис этого пространства в данной точке $p$ (которая задается локальными координатами $\zeta$).Вообще говоря, вектора $\mathbf{i}_k$ различны в каждых точках пространства и не обязаны иметь единичную длину или свойство ортогональности. Обозначим скалярное произведение таких векторов:
$$g_{ij} = (\mathbf{i}_i, \mathbf{i}_j)$$
Тензор $g_{ij}$ называется метрическим тензором.

Разложим аналогично с декартовой системой произвольные вектора $\mathbf{V,W}$ по базису $\mathbf{i}$. Для скалярного произведения получим:
$$(\mathbf{V,W)} = (\mathbf{i}_i, \mathbf{i}_k) V^i W^k = g_{ik} V^i W^k = V_i W^i,$$
где в последнем равенстве мы определили \textit{ковариантные} компоненты $V_i = g_{ij} V^j$ вектора $\mathbf{V}$ ($V^k$ называются \textit{контравариантными} компонентами). Такое определение согласуется с известным вам определением ковариантных и контравариантных компонент из курса линейной алгебры. Проводя дальнейшую аналогию, ковариантные компоненты $V_k$ мы можем рассматривать как  компоненты некоторой линейной формы (линейного функционала). Действие такой 1-формы на произвольной вектор (присоединенный к той же точке) представляет собой скалярное произведение:
$${V}(\cdot) = (\mathbf{V}, \cdot)$$.

Все линейные формы присоединенные к точке $\zeta^i$ образуют векторное пространство (двойственное к пространству векторов).   

В дальнейшем мы будем фокусироваться на системах криволинейных координат где метрический тензор имеет простой (диагональный) вид:
$$(\mathbf{e}_i, \mathbf{e}_j) = h^2_i \delta_{ij}$$ 
Коэффициенты $h_i$, нормы базисных векторов, носят название коэффициентов Ламе.
В таких системах можно ввести ортонормированный базис\footnote{\textbf{Важно}: зависимость векторов $e_i$ от точки пространства не исчезает}.:
$$ \mathbf{e}_i = \frac{\mathbf{i}_i}{(\mathbf{i}_i,\mathbf{i}_i)}$$


В таком базисе скалярное произведение выглядит особенно просто:
$$(\mathbf{V}, \mathbf{W}) = \bar{V}^k \bar{W}^k = \bar{V}_k \bar {W}^k,$$
где $\bar{x}^k$ - разложения векторов по ортонормированному базису. Это обозначение мы будем также использовать и далее.

Рассмотрим некоторую функцию $S$, определенную в каждой точке нашего пространства(на нашем многообразии). В каждой точке полный дифференциал имеет вид:
$$dS = \frac {\partial S}{\partial \zeta^i} d \zeta^i$$
и является линейной формой, а точнее, дифференциальной 1-формой. Как и для всякой линейной формы(линейного функционала) аргументом выступает элементы ВП, а областью определения - поле скаляров этого векторного пространства.
Дифференциалы $d \zeta^i$ образуют базис на пространстве 1-форм, действие форм на любой вектор определяется соотношением на базисных векторах:
$$d \zeta^k (\mathbf{i}_j) = \delta^k_j$$ 
и линейностью для продолжения. Частные производные $ \partial_{\zeta^i} S$ являются компонентами 1-формы $dS$ в естественном базисе $d \zeta^i$.

На векторе $\delta \boldmath{\zeta}= i_k \delta \zeta^k$ - малом смещении $\delta \zeta^i$ полный дифференциал действует следующим образом:
$$dS (\delta \mathbf{\zeta}) = \frac {\partial S}{\partial \zeta^i} d \zeta^i (i_k \delta \zeta^k) = \frac {\partial S}{\partial \zeta^i} \delta \zeta^i \approx S(\zeta + \delta\zeta) - S(\zeta)$$
Т.е. $dS(\delta \zeta)$ и есть $dS$ в привычном понимании.

В системах с диагональным метрическим тензорам можно ввести другой базис 1-форм, определив его действием на единичные базисные вектора:
$$f^k (\mathbf{e_j}) = \delta^k_j$$
Откуда в силу линейности:
$$f^k = h_k d \zeta^k$$ 
Действие линейной формы $\hat{w} = w_k d \zeta^k$ в точке $\zeta$ на вектор $\mathbf{V}$, присоединенный к той же точке $\zeta$, задается следующим образом:
$$ \hat{w} (\mathbf{V}) = w_k d \zeta^k (\mathbf{i}_j V^j) = w_k V^k = h_k \bar{w}_k h^{-1}_k \bar{V}^k= \bar{w}_k \bar{V}^k $$
Это показывает что компоненты 1-формы таким образом можно рассматривать как ковариантные компоненты некоторого вектора, а действие формы задавать через скалярное произведение с этим вектором $\mathbf i_i w^i = \mathbf{e_i} \hat{w}^i$.

\textit{Градиентом функции} $S$ будем называть полный дифференциал $dS$ отнесенный к базису $f^k$:
$$dS = \frac{\partial S}{\partial \zeta^k} d \zeta^k= (\frac{1}{h_k} \frac{\partial S}{\partial \zeta^k}) f^k = \overline{\nabla S}_k f^k$$
Градиент функции $S$, или другими словами полная производная $S$, это линейный функционал векторов касательного пространства $T_p M$ в $\mathbb{R}$:
$$dS(V) = V^l \frac{\partial S}{\partial \zeta^k} d \zeta^k (\mathbf{i}_l) = V^k \frac{\partial S}{\partial \zeta^k}$$

Определим некоторые абстрактные вещи, смысл которых будем прояснять постепенно.

\textit{Внешней алгеброй} ВП $V$ называется алгебра порожденная элементами $1, e_i$, со следующими определяющими соотношениями:
\begin{itemize}
    \item$ e_i \wedge e_j = - e_j \wedge e_i$
    \item $e_i \wedge 1 = e_i$
\end{itemize}

 \textit{Дифференциальной} $k$\textit{-формой} на $\mathbb{R}^n$ будем  называть объекты вида:
$$\omega = \sum_{1\leq i_1 < i_2 <\dots < i_k \leq n} f_{i_1, i_2, \dots, i_k} (x^1, \dots) dx^{k_1} \wedge dx^{k_2} \wedge \dots \wedge d x^{i_k}$$, 
где $f$ - гладкие функции, $dx^i$ - дифференциалы координаты $x^i$. При смене базиса это представление меняет форму.

\textit{Внешней производной} для формы\footnote{Здесь $I$ - мультииндекс, и под $dx^I$ поднимается внешнее произведение всех дифференциалов}
$$\phi = g dx^I $$
называется форма:
$$ d\phi = \frac{\partial g}{\partial x^i} dx^i \wedge dx^I$$.



\textit{Оператор Ходжа} $*$ переводит формы ранга $k$ в формы ранга $n-k$.
Сначала определим действие на произвольном внешнем произведении:
$$* (dx^{i_1} \wedge \dots \wedge d x^{i_k}) = \frac{\sqrt{g}}{(n-k)!} g^{i_1 j_1} \dots g^{i_k j_k} \epsilon_{j_1 \dots j_n} dx^{j_{k+1}} \wedge \dots \wedge dx ^ {j_n}$$
Пусть $\alpha$ - произвольная форма, раскладывая её в нашем базисе:
$$ \alpha = \frac{1}{k!} \alpha_{i_1, \dots i_k} dx^{i_1} \wedge \dots \wedge dx^{i_k}$$
и обозначив\footnote{Важно: индексы подняты}:
$$ \beta_{i_{k+1} \dots i_n}= \frac{\sqrt{g}}{k!} \alpha^{i_1\dots i_k} \epsilon_{i_1,\dots, i_n}$$
определим оператор следующим образом:
$$(*\alpha) = \frac{1}{(n-k)!} \beta dx^{i_{k+1}} \wedge \dots \wedge dx^{i_n}$$

Займемся теперь определением дивергенции векторного поля $\mathbf{V} = i_k V^k$. Рассмотрим 1-форму $V$, соответствующую векторному полю$\mathbf{V}$:
$$V = V_i d \zeta^i = g_ik V^k d \zeta^i,$$
применим к ней оператор Ходжа:
$$*V = \frac{1}{2} \sqrt{g} \epsilon_{ijk} V^k d \zeta^i \wedge d \zeta^j,$$
где $g = det (g_ij)$ и  возьмем внешнюю производную от получившейся 2-формы:
$$d(*V) = \frac{1}{2}  \epsilon_{ijk} \frac{\partial}{\partial \zeta^l} 
\left(V^k \sqrt{g}\right) d \zeta^l \wedge d \zeta^i \wedge d \zeta^j$$
Отметив, что:
$$ d\zeta^l \wedge d \zeta^i \wedge d \zeta^j = \epsilon_{lij} d \zeta^1 \wedge d \zeta^2 \wedge d \zeta^3,$$
получим:
$$d(*V) = \frac{\partial}{\partial \zeta^l} 
\left(V^k \sqrt{g}\right) d \zeta^1 \wedge d \zeta^2 \wedge d \zeta^3$$
Полученная 3-форма $d(*V)$,отнесенная к канонической форме объема $f^1 \wedge f^2 \wedge f^3$ является \textit{дивергенцией векторного поля V}:
$$ \mathbf{div} \mathbf{V} = \frac{1}{h_1 h_2 h_3} \partial_{\zeta^k} (h_1 h_2 h_3 \frac{\hat{V}^k}{h_k})$$ 

Поменяв порядок операций, построим циркуляцию векторного поля. Сначала возьмем внешнюю производную:
$$dV = \frac{\partial}{\partial \zeta^k} \left(g_{ij} V^j\right) d \zeta^k \wedge d \zeta^i,$$
и к полученной 2-форме применим оператор Ходжа:
$$*(dV) = \sqrt{g}g^{kl}g^{im} \frac{\partial}{\partial \zeta^k} (g_{ij} V^j) \epsilon_{lmn} d \zeta^n$$
В системах Ламе, компоненты полученной 1-формы в базисе $f^i$ совпадают с обычным определением циркуляции:
$$*(dV) = h_1 h_2 h_3 h^{-2}_k h^{-2}_i \epsilon_{kin} \frac{\partial}{\partial \zeta^k} (h_i V^i) h^{-1}_n f^n = \overline{[\mathbf{\nabla}, \cross \mathbf{V}]}f^n$$
После упрощения:
$$\overline{[\mathbf{\nabla}, \mathbf{V}]} = \frac{h_n}{h_1 h_2 h_3} \epsilon_{kin} \partial_{\zeta^k} (h_k \hat{V}^k)$$.

\subsection*{Пример}
Рассмотрим сферическую систему координат $(r, \theta, \phi)$, введенную обычным образом:
\begin{align*}
    x & = r \sin \theta \cos \phi \\
    y &= r \sin \theta \sin \phi \\
    z &= r \cos \theta \\
\end{align*}
Построим базисные вектора:
$$ 
\mathbf{i}_r = \begin{pmatrix}
\sin \theta \cos \phi \\
\sin \theta \sin \phi \\
\cos \theta\\
\end{pmatrix},
\mathbf{i}_{\theta} =\begin{pmatrix}
r \cos \theta \cos \phi \\
r \cos \theta \sin \phi \\
-r \sin \theta\\
\end{pmatrix},
\mathbf{i}_{\phi} =\begin{pmatrix}
- r \sin \theta \sin \phi \\
r \sin \theta \cos \phi \\
0\\
\end{pmatrix};
$$
и вычислим по ним коэффиценты Ламе. $h_r = 1. h_\theta = r, h_\phi = r \sin \theta$.
Каноническая форма объема тогда примет следующий вид:
$$f^r \wedge f^\theta \wedge f^\phi = r^2 \sin \theta dr \wedge d \theta \wedge d \phi$$.

\newpage
\subsection*{Литература}
Базовая:
\begin{itemize}
    \item Зорич В. А., Математический анализ, Части II.
    \item Арнольд В. И., Математические методы классической механики. 
    \item  W. Rudin, Principles of mathematical analysis
\end{itemize}
Продвинутая:
\begin{itemize}
    \item С. П. Новиков, И. А. Тайманов, Современные геометрические структуры и поля.
    \item Картан А., Дифференциальное исчисление. Дифференциальные формы.
    \item H. Flanders, Differential forms and applications to the physical sciences
    \item M. Spivak, Calculus on manifolds
    \item Bishop, R.; Goldberg, S. I. (1980), Tensor analysis on manifolds
\end{itemize}
\newpage
\centering{
\subsection*{Листок - 2}}
\begin{enumerate}
    \item Найдите значения $\omega$ на указанных векторах:
    \begin{enumerate}
        \item $\omega = x^2 dx^1$ на векторе $\zeta = (1,2,3)$
        \item $\omega = d x^1 \wedge d x^3 + x^1 dx^2 $ на паре векторов $\zeta_1, \zeta_2$
    \end{enumerate}
    \item Образуют ли формы ранга $k$ векторное пространство? Какой размерности?
    \item Пользуясь определением $\wedge$ покажите, что для форм $\omega, \eta$ рангов $k,l$ верно $\omega \wedge \eta = (-1)^{kl} \eta \wedge \omega$
    \item Любая ли 1-форма является дифференциалом? 
    \item Как преобразуются координаты 1-форм при гладких замене координат?
    \item Покажите что для внешнего произведения $\omega^1 \wedge \omega^1 = 0$. Верно ли это для $k>1$?
    \item Вычислите коэффициенты Ламе для параболической системы координат. Выпишите в явном виде операторы дивергенции, ротора, градиента
    \item Вычислите коэффициенты Ламе для цилиндрической системы координат. Выпишите в явном виде операторы дивергенции, ротора, градиента.
    \item Под оператором Лапласа $d(*dS)$ понимаем дивергенцию градиента. Получите общий вид в системах Ламе, предъявите явную запись оператора в сферических координатах.

\vspace{\fill}
\textit{Задача 9 - обязательная. Листочек считается сданным если сдано 5 задач из 8 и 9 задача в срок до 1 ноября 2018}

\end{enumerate}
\end{document}

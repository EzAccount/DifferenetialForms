\documentclass{article}
\usepackage{amsmath,amsthm,amssymb}
\usepackage[T1,T2A]{fontenc}
\usepackage[utf8]{inputenc}
\usepackage[english,russian]{babel}
\usepackage{fancyhdr}
\usepackage[a5paper]{geometry}
\pagestyle{fancy}
\fancyhf{}
\begin{document}
Множество G с заданной на нем бинарной операцией $*: G \times G \to G$ называется \textit{группой} $(G, *)$ если:
\begin{enumerate}
    \item $\forall a,b,c \in G: a*(b*c) = (a*b)*c $
    \item $\exists e \in G: \forall a \in G, a*e=e*a=a$
    \item $\forall a \in G, \exists b: a*b = e$
\end{enumerate}


\subsection*{Доказать}

\noindent0. Матрицы вида  $\left(\begin{matrix}a & b\\ -b & a\end{matrix}\right) a,b \in \mathbb{R}$ образуют группу относительно стандартного умножения матриц ($c_{ij} = a_{ik} b_{kj}$).

\noindent1. Множество векторов ВП образуют группу относительно сложения.

\noindent2. Определитель кососимметричной матрицы нечетного размера равен нулю.

\noindent3. Пусть C = $\left(\begin{matrix}A & X\\ 0 & B\end{matrix}\right) A,X, B \in \mathbb{R}^{n \times n}$. Докажите, что $\det С = \det A \det B$

\noindent4. $\det \text{adj} A = (\det A)^{n-1}$ [adj - присоединенная матрица]
\subsection*{Вычислить}

\noindent5. $\det A$, если

 a) $a_{ij} = \min(i,j)$
 
 b) $a_{ij} = \max(i,j)$ 
 
\noindent6.\ $ \begin{vmatrix}
    1 & -1 &    &    & 0   \\
    -1  & 1       & -1   &      &    \\
      &       \ddots &           \ddots&     \ddots      &    \\
      &         &           -1&          1 & -1\\
        0  &         &           &          -1 & 1
  \end{vmatrix}
  $
  
\noindent7.\ $\partial_{ij} \det A$


\end{document}
\textsl{}
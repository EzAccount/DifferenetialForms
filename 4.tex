\documentclass{article}
\usepackage{amsmath,amsthm,amssymb}
\usepackage{amsthm}
\usepackage{mathtools}
\usepackage{graphicx}
\usepackage{wrapfig}
\graphicspath{ {images/} }
\usepackage[T1,T2A]{fontenc}
\usepackage[utf8]{inputenc}
\usepackage[english,russian]{babel}
\usepackage{fancyhdr}
\usepackage[a4paper]{geometry}
\pagestyle{fancy}
\fancyhf{}
\fancyhead[L]{Когомологии де Рама.}
\theoremstyle{definition}
\newtheorem{defn}{Def}
\newtheorem{example}{Ex}
 \newtheorem{theorem}{Th}
 \DeclareMathOperator{\im}{im}
\begin{document}
\section*{Напоминания из прошлой лекции}
\begin{example} Внешнее произведение форм $\omega = x^2 dx^1 \wedge dx^3 + dx^2 \wedge dx^4$ и $\eta = (x^1+1)dx^2 \wedge dx^4$
\begin{multline*}
\omega \wedge \eta = x^2 (x^1+1)dx^1 \wedge dx^3 \wedge dx^2 \wedge dx^4 + (x^1+1)dx^1\wedge dx^2 \wedge dx^2 \wedge dx^4 =\\=
-x^2(x^1+1) dx^1 \wedge dx^2 \wedge dx^3 \wedge dx^4
\end{multline*}
\end{example}
\begin{example}
	Внешний дифференциал формы $\omega = (x^1+x^3 x^3) dx^1 \wedge dx^2$:
	\begin{equation*}
	d\omega = dx^1 \wedge dx^1 \wedge dx^2 + 2x^3 dx^3\wedge dx^1 \wedge dx^2 = 2x^3 dx^1\wedge dx^2 \wedge dx^3
	\end{equation*}
\end{example}
\section*{Интегрирование форм}
Параметризуем $k-$многообразие $\Omega_k \subset R^d$ с помощью $\tau^1, \dots, t^k:$
\begin{gather}
	\xi^1 = \xi^1(\tau^1, \dots, \tau^k)\\	
	\dots \\
	\xi^d = \xi^d(\tau^1, \dots, \tau^k)\\
\end{gather}
Для $\Omega_k$ при такой параметризации существует касательное пространство со следующим базисом:
\begin{equation*}
\mathbf{t}_i = \mathbf{i}_1 \frac{\partial \xi^1}{\partial \tau^i} + \dots + \mathbf{i}_d \frac{\partial \xi^d}{\partial \tau^i}
\end{equation*}
Вектор $\mathbf{t}_i$ - касательный к кривой на $\Omega_k$, полученный варьированием одного параметра $\tau_i$ при фиксированных остальных. Выбор порядка векторов определяет относительную(по отношению к исходному многообразию) ориентацию.
\begin{defn}
	\begin{equation}
	\int_{\Omega_k}  \ ^{(k)} \omega = \int d \tau^1 \dots \int d \tau^k \omega(\mathbf{t}_1, \dots \mathbf{t}_k)
	\end{equation}
\end{defn}


	
	\begin{example}Рассмотрим 1-мерное многообразие $M$ вложенное в $\mathbb{R}^3$, параметризованное следующим образом:

 $$M : (3t, t^2, 5-t), t \in [0;2]$$
и заданную 1-форму: $\omega = 2x^2 dx^1 - x^1 x^3 dx^2 + dx^3$.
	
	Вычислим интеграл от этой формы по кривой:
	
	\begin{equation*}
	\int_M \omega = \int_0^2 \begin{pmatrix} 3 \\ 2t \\ -1 \end{pmatrix} dt = \int_0^2 (2x^2\cdot 3 - x^1 x^3\cdot 2t - 1) dt = \int_0^2(6t^3-24t^2-1)dt
	\end{equation*}
\end{example}
\begin{theorem}[Stockes]
	\begin{equation}
	\int_{\Omega_{p+1}} d(\ ^{(p)} \omega) = \int_{\partial\Omega_{p+1}} \ ^{(p)}\omega
	\end{equation}
\end{theorem}
\begin{example}
	Рассмотрим 2-многообразие $M$ вложенное в $\mathbb{R}^4$ следующим образом:	
	\begin{gather}
	M = (u^1, u^1 - u^2, 3 - u^1+u^1u^2, -3u^2)
	u^1u^1+u^2+u^2<1
	\end{gather}
	
Граница многообразия - кривая, описываемая одним параметром $t$. Граница параметризуется в $\mathbb{R}^4$ следующим образом:
\begin{equation}
	\partial M = (\cos t, \cos t - \sin t, 3 - \cos t+ \cos t \sin t, -3 \sin t)
\end{equation}
Рассмотрим форму $\omega$:
\begin{gather}
 \omega = x^3 x^3 dx^1
 d \omega = -2x^2 dx^1 \wedge d x^3
\end{gather}
и убедимся в правильности теоремы Стокса. Начнём с вычисления интеграла по многообразию от дифференциала. Вычислим базис в касательном пространстве:
\begin{equation}
t_1=\begin{pmatrix} 1 \\ 1 \\ -1 + u_2 \\ 0\end{pmatrix} \hspace{1cm} t_2 = \begin{pmatrix}
0\\ -1 \\ u_1 \\-3
\end{pmatrix}
\end{equation}
\begin{equation}
dx^1 \wedge dx^3 (t_1, t_2) = det \begin{vmatrix} 1 & 0 \\ u^2-1 & u_1\end{vmatrix}
\end{equation}

\begin{equation}
\int_M d \omega = \int_{-1}^{1} \int_{-\sqrt{1-u_1^2}}^{\sqrt{1+u_1^2}} -2(3-u_1+u_1u_2)u_1 du_2 du_1 = \frac{\pi}{2}
\end{equation}
С другой стороны, вычислим касательный вектор к кривой $\partial M$
\begin{equation}
m = \begin{pmatrix} -\sin t \\ -\sin t - \cos t \\ \sin t + \cos 2t \\ -3 cos t\end{pmatrix}
\end{equation}
Тогда вычисляя интеграл от формы по границе многообразия:

\begin{equation}
\int_{\partial M} \omega = \int_0^{2\pi} \omega(m) dt = \int_0^{2\pi} (3-\cos t + \cos t \sin t)^2 (-\sin t)dt = \frac{\pi}2
\end{equation}
\end{example}

\end{document}